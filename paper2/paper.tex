\documentclass[uplatex]{sumiilab-paper}

\usepackage{amsmath,amssymb} % 数式
\usepackage{bm}
\usepackage{mathtools} % 数学記号

\usepackage{listings, jvlisting} % ソースコード
% \ttfamilyで太字を使うためのおまじない(\colorでのハイライトにする場合は不要)
\usepackage[T1]{fontenc}
\usepackage[scaled=1.0]{beramono}

\usepackage{bussproofs} % 証明器
\usepackage{amsthm} % 定理環境

\usepackage{enumitem} % リスト環境
\usepackage{cite} % 引用
\usepackage[dvipdfmx]{graphicx} % 各種形式の画像を簡単にincludeできる
\usepackage[dvipdfmx, hidelinks, setpagesize=false]{hyperref} % リンク機能強化
\usepackage{pxjahyper} % しおりの日本語対応


%% =========================================
%% 定理環境の設定
%% =========================================
\newtheoremstyle{mystyle}% name
{}% space above
{}% space below
{\normalfont}% body font
{}% indent amount
{\bfseries}% theorem head font
{ }% punctuation after theorem head
{4pt}% space after theorem head (default: 5pt)
{\thmname{#1}\thmnumber{#2}\thmnote{\hspace{2pt}(#3)}}% theorem head spec

\theoremstyle{mystyle}
\newtheorem{definition}{定義}
\newtheorem{theorem}[definition]{定理}
\newtheorem{corollary}[definition]{系}
\newtheorem{proposition}[definition]{命題}
\newtheorem{lemma}[definition]{補題}
\newtheorem{example}[definition]{例}
\newtheorem{assumption}[definition]{仮定}
\newtheorem{axiom}[definition]{公理}
\renewcommand{\proofname}{\bf{証明}}
\numberwithin{definition}{chapter} % 定義1.1のように表示

%% ===============================================
%% ソースコードの設定
%% ===============================================
% プログラミング言語と表示するフォント等の設定
\lstset{
  language={[Objective]Caml},% プログラミング言語
  basicstyle={\ttfamily\small},% ソースコードのテキストのスタイル
  keywordstyle={\bfseries},% 予約語等のキーワードのスタイル
  commentstyle={},% コメントのスタイル
  stringstyle={},% 文字列のスタイル
  frame=tb,% ソースコードの枠線の設定 (none だと非表示、trlbだと全方位に枠がつく)
  numbers=left,% 行番号の表示 (none だと非表示)
  numberstyle={\footnotesize},% 行番号のスタイル
  xleftmargin=15pt,% 左余白
  xrightmargin=5pt,% 右余白
  keepspaces,% 空白を維持する
  mathescape,% $ で囲った部分を数式として表示する ($ がソースコード中で使えなくなるので注意)
  % 手動強調表示の設定
  moredelim=[is][\bfseries]{@*}{*@},
  moredelim=[is][\itshape]{@/}{/@}
}
% インライン表示用の環境設定
\lstnewenvironment{code}{\lstset{
  frame=none,
  numbers=none
}}{}
% 本文中にコードを|foo|の形式で書くことができる
\lstMakeShortInline[columns=fullflexible]|

%% ===============================================
%% 論文中で使う記号とかのマクロ定義
%% ===============================================

%% 論文中で繰り返し使う記号は次のように「マクロ」として実装しておくと良い。
%% TeX ソース中で \BOOL と書くと、\texttt{Bool} に置き換えてくれる。
%% フォントを変え忘れたりするリスクが減るし、あとから記号を変更するのも楽になる。

\newcommand{\bkeyword}[1]{\ensuremath{\mathbf{#1}}}
\newcommand{\BOOL}{\bkeyword{Bool}}
\newcommand{\TRUE}{\bkeyword{true}}
\newcommand{\FALSE}{\bkeyword{false}}
\newcommand{\IF}{\bkeyword{if}}
\newcommand{\THEN}{\bkeyword{then}}
\newcommand{\ELSE}{\bkeyword{else}}

%% ===============================================
%% 論文の表紙に表示される情報
%% ===============================================

% 論文の年度と種類
\paper{20XX 年度 卒業論文}% 学部生
%\paper{20XX 年度 修士論文}% 修士

% 論文のタイトル
\title{住井研究室の\\ステキな論文クラスファイルの使用例}

% 学籍番号と著者のお名前
\author{X0XX1234 ラムダ 小太郎}

% 著者の所属
\institute{東北大学 工学部\\電気情報物理工学科}% 学部生
%\institute{東北大学 大学院情報科学研究科\\情報基礎科学専攻}% 修士

% 指導教員のお名前
\supervisor{住井 英二郎 教授}% 指導教員
\subsupervisor{松田 一孝 准教授}% 論文指導教員(「論文指導教員=指導教員」の場合は省略する)

% 論文発表日時
\date{20XX 年1月1日 \quad 23:00--23:30}
% 発表場所
\venue{電子情報システム・応物系1号館2階トイレ}
%% 体裁にこだわりたい人向け情報:
%% 「論文発表日時」と「発表場所」は利便性のためなので、学位論文には埋め込まない方が一般的。
%% こだわりたい人はこれらを学位論文表紙から除去し、スライドや教員へのメールに記載すればいい。
%% (sumiilab-paper.clsは微妙にダサいですが、特に学部生は時間がないので気にせずそのまま使う方がいいかと。)


\begin{document}
\frontmatter% ここから前文

\maketitle

\begin{abstract}
ステキな論文の概要
\end{abstract}

\tableofcontents% 目次

\mainmatter% ここから本文

\chapter{序論}

貢献、構成までしっかり書きます。
%% 体裁にこだわりたい人向け情報:
%% 「、。」は好みですが、数式を沢山書くなら「,.」にした方が無難かと。
%% (あらゆる教科書がそうなっていると思います。)

%% 参考文献は\cite{ID}します(IDはrefs.bib内で文献につけた識別子)
%% BibTeXの使い方などは各自調べて下さい。
\cite{Pierce:TypeSystems}

%% 一方labelは\refします。
第\ref{c:how-to-use-tex}章では、。。。

\chapter{\TeX の簡単な使い方}
\label{c:how-to-use-tex}

\ref{s:source-code}節では、。。。
\ref{ss:example-theorem}項では、。。。

\section{ソースコード}
\label{s:source-code}

図\ref{src:listup_nodes}は二分木を深さ優先探索してノードを列挙する関数である。
\begin{figure}[t]
\begin{lstlisting}[label=src:listup_nodes]
type 'a bin_tree =
  | Leaf of 'a
  | Node of 'a bin_tree * 'a bin_tree

let rec listup_nodes = function
  | Leaf x -> [x]
  | Node (r, l) -> (listup_nodes r) @ (listup_nodes l)
\end{lstlisting}
\caption{二分木のノードのリストアップ}
\end{figure}

図ではなく、インラインに書く場合もある。
二分木を深さ優先探索してノードを列挙する関数|listup_nodes|を次に示す。
\begin{code}
type 'a bin_tree =
  | Leaf of 'a
  | Node of 'a bin_tree * 'a bin_tree

let rec listup_nodes = function
  | Leaf x -> [x]
  | Node (r, l) -> (listup_nodes r) @ (listup_nodes l)
\end{code}


\section{画像}

画像の貼り方については、docs/EPSIMAGES.mdが推奨している|convert|ではなく、graphicxから扱う方が基本的によい。
\footnote{
もしgraphicxが標準で入っていない場合は、\verb|tlmgr install graphicx|などでインストールすると良い。
}
図\ref{f:aaa}はdblp\_bibtex\_crossrefである。
% ちなみに、初めて言及されたページのトップ(\url{https://chi2014.acm.org/templates/SIGCHIpaperformat.pdf})か、
% その次のページのトップ(\url{https://www.acm.org/binaries/content/assets/publications/taps/acm_layout_submission_template.pdf})
% に貼り付けるのが慣例だと思う。

\begin{figure}[t]
  \centering
  \includegraphics[width=.98\linewidth]{../docs/dblp_bibtex_crossref.png}
\caption{試しに貼り付けられたdblp\_bibtex\_crossref}
\label{f:aaa}
\end{figure}

\section{BNFの書き方の例}

本節では、BNFによるプログラミング言語の構文の書き方を紹介する。
構文木の書き方は一つというわけではないので、幾つかのバリエーションを紹介する。
どの方法が良いと思うかは、個人の好みに依るところなので、好きなものを使えば良いと思う。

まず、次の方法では、array環境を使って、BNFを書いている。
array環境は数式環境中で表のようなものを書くときに使う。
基本的に、table環境と使い方は同じである。
\[
%% 空白を明示的に開けるときは "\," "\ " "~" "\quad" "\qquad" などを使う。
%% 空白の幅は "\qquad" > "\quad" > "~" = "\ " > "\," の順で大きい。
%% "~" と "\ " は空白の代わりに改行を許すかどうかの違い("\ " だと改行される可能性がある)
\begin{array}{rcl@{\qquad\qquad}r}
  t & \Coloneqq & & \text{terms:} \\
  & \mid & x & \text{variables} \\
  & \mid & \lambda x.~t & \text{lambda abstraction} \\
  & \mid & t_1~t_2 & \text{application} \\
  & \mid & \TRUE & \text{true} \\
  & \mid & \FALSE & \text{false} \\
  & \mid & \IF~t_1~\THEN~t_2~\ELSE~t_3 & \text{if statement}
\end{array}
\]

他にも、次のように、align環境を使っても、似たようなものを書くことができる。
\begin{align}
  t \Coloneqq & \tag*{terms:} \\
  {}\mid{} & x \tag*{variables} \\
  {}\mid{} & \lambda x.~t \tag*{lambda abstraction} \\
  {}\mid{} & t_1~t_2 \tag*{application} \\
  {}\mid{} & \TRUE \tag*{true} \\
  {}\mid{} & \FALSE \tag*{false} \\
  {}\mid{} & \IF~t_1~\THEN~t_2~\ELSE~t_3 \tag*{if statement}
\end{align}
array環境を愚直に使う場合と比べて、式が中央揃えになるという点と、
``variables''とかの説明が右端に来ている点が違う。
説明はtag*マクロで出しており、これはもともと式番号を指定するためのものなので、
若干使い方がおかしい気もするが、まぁ、いいだろう。
自分の好みの方を使うと良いだろう。

BNF全体を左揃えにしたいならば、次のように、flalign環境を使うと良い。
align環境と違って、\verb|&|を余分に1つ付ける必要がある、ということに注意して欲しい(詳しくはソースコードを見よ)。
\begin{flalign}
  t \Coloneqq & & \tag*{terms:} \\ % & を余分に1つ付けること!
  {}\mid{} & x \tag*{variables} \\
  {}\mid{} & \lambda x.~t \tag*{lambda abstraction} \\
  {}\mid{} & t_1~t_2 \tag*{application} \\
  {}\mid{} & \TRUE \tag*{true} \\
  {}\mid{} & \FALSE \tag*{false} \\
  {}\mid{} & \IF~t_1~\THEN~t_2~\ELSE~t_3 \tag*{if statement}
\end{flalign}

\section{導出木の書き方の例}

導出木の書き方も色々あるが、ここでは、bussproofs.sty を使った方法を紹介する。
導出木は、手書きでも書きにくいが、\LaTeX だから書きやすいというわけでもなく、
(使うパッケージにも依るが)そこそこの苦労は必要である。
bussproofs.sty を除く多くの方法では、frac などをベースに「分数」で導出木を書く。
bussproofs.sty はこれらとは全く異なるインタフェースであり、慣れれば比較的解りやすい。
bussproofs.sty の動作は、(導出木を要素とする)スタックをイメージすると解りやすい。
よく使うマクロは次の通り。
\begin{itemize}
\item \verb|\AxiomC{...}|:Axiom を push する(導出木では葉に相当)
\item \verb|\UnaryInfC{...}|:スタックから部分導出木(仮定)を1つ pop して、
  それを新たに作ったノード(結論)の子供にすることで、新たな部分導出木を作成し、push する。
\item \verb|\BinaryInfC{...}|:スタックから部分導出木(仮定)を2つ pop して、
  \verb|\UnaryInfC| と同様の動作を行う。
\item \verb|\TrinaryInfC{...}|:スタックから部分導出木(仮定)を3つ pop して、
  \verb|\UnaryInfC| と同様の動作を行う。
\end{itemize}

実際の使い方は以下の通り。

%% T-Var
\begin{prooftree}
  \AxiomC{$x:T \in \Gamma$}
  \RightLabel{\textsc{T-Var}}
  \UnaryInfC{$\Gamma \vdash x : T$}
\end{prooftree}
%% T-Abs
\begin{prooftree}
  \AxiomC{$\Gamma, x:T \vdash t : U$}
  \RightLabel{\textsc{T-Abs}}
  \UnaryInfC{$\Gamma \vdash \lambda x.~t : T \to U$}
\end{prooftree}
%% T-App
\begin{prooftree}
  \AxiomC{$\Gamma \vdash t_1 : T \to U$}
  \AxiomC{$\Gamma \vdash t_2 : T$}
  \RightLabel{\textsc{T-App}}
  \BinaryInfC{$\Gamma \vdash t_1~t_2 : U$}
\end{prooftree}

\begin{prooftree}
  \AxiomC{}
  \RightLabel{\textsc{T-True}}
  \UnaryInfC{$x : \BOOL \to \BOOL \vdash \TRUE : \BOOL$}
  \RightLabel{\textsc{T-Abs}}
  \UnaryInfC{$\vdash \lambda x.~\TRUE : (\BOOL \to \BOOL) \to \BOOL$}
  \AxiomC{$y : \BOOL \in y : \BOOL$}
  \RightLabel{\textsc{T-Var}}
  \UnaryInfC{$y : \BOOL \vdash y : \BOOL$}
  \RightLabel{\textsc{T-Abs}}
  \UnaryInfC{$\vdash \lambda y.~y : \BOOL \to \BOOL$}
  \RightLabel{\textsc{T-App}}
  \BinaryInfC{$\vdash (\lambda x.~\TRUE)~(\lambda y.~y) : \BOOL$}
\end{prooftree}

\section{定理環境}

amsthm.styをカスタマイズした定理環境を使う。

\begin{theorem}[定理のタイトル]
  定理の内容
\end{theorem}

\begin{lemma}[補題のタイトル]
  補題の内容
\end{lemma}

\begin{corollary}[系のタイトル]
  系の内容
\end{corollary}

\begin{proposition}[命題のタイトル]
  命題の内容
\end{proposition}

\begin{definition}[定義のタイトル]
  定義の内容
\end{definition}

\begin{example}[例のタイトル]
  例の内容
\end{example}

\begin{assumption}[仮定のタイトル]
  仮定の内容
\end{assumption}

\begin{axiom}[公理のタイトル]
  公理の内容
\end{axiom}

\begin{proof}
  証明の内容
\end{proof}

\subsection{定理環境の使い方の例}
\label{ss:example-theorem}

\begin{lemma}
  \label{lem:interesting-lemma}
  論文の中で最重要とは言えないような性質・命題は補題(lemma)にする。
  補題や定理から直ちに導けるような軽い命題は系(corollary)にする(細かい使い分けは人による)。
\end{lemma}

\begin{proof}
  \lstinline|proof*| のように、アスタリスク付きの環境では、番号が付かない。
\end{proof}

\begin{theorem}
  \label{thm:wonderful-theorem}
  提案手法の最も重要な性質や命題は、定理(theorem)として書く。
  読者の心をくすぐる興味深いステートメントを書こう。
\end{theorem}

\begin{proof}
  定理 \ref{thm:wonderful-theorem} の華麗な証明。その美しい証明に、読者の目は釘付けだ!
  \begin{enumerate}[leftmargin=0pt,itemindent=*,label=Case \arabic*.]
  \item 自明
  \item 補題 \ref{lem:interesting-lemma} から直ちに導ける。
  \item 言うまでもない。目を瞑れば証明が見えてくる。
  \item あんまり自明じゃない
    \begin{enumerate}[label=(\roman*)]
    \item 自明じゃないと思ったけど、やっぱり自明だった
    \item ほらね、こんなに簡単
    \end{enumerate}
  \end{enumerate}
\end{proof}

\chapter{関連研究}
\label{c:related}

\chapter{結論}
\label{c:conclusion}

まとめと今後の課題。

\backmatter% ここから後付
%% 体裁にこだわりたい人向け情報:学位論文の謝辞は概要の後につけた方がいい。
\chapter{謝辞}

ステキな謝辞
% 学位論文の謝辞は独特のマナーがあるので他人の修論を参考にしたほうがいいかも:
% \url{https://tohoku.repo.nii.ac.jp/search?page=1&size=20&sort=-createdate&search_type=2&q=81}
% (適当に書いても問題があれば指導教員に指摘されるだけなので、その時直せば十分という説もあり。)

%% 参考文献: bibtex
\bibliographystyle{jplain}
\bibliography{refs}

\appendix% ここから付録
\chapter{付録}
(必要に応じて)ステキな付録
%% 体裁にこだわりたい人向け情報:付録の\labelを\refしたければ、\backmatterを消せばよかったはず。
\end{document}
